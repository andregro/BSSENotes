\documentclass[a4paper,11pt,twocolumn]{report}
\usepackage{geometry}
\geometry{left=2cm,right=2cm,bottom=2.5cm,top=2cm}
\usepackage[utf8]{inputenc}
\usepackage[T1]{fontenc}
\usepackage{color}
\usepackage{graphicx}
\DeclareGraphicsExtensions{.pdf,.png,.jpg}
\graphicspath{ {images/} }
\setcounter{tocdepth}{1}
\title{Summary to\\"Pattern-oriented software architecture Volume 1:\\
    A system of patterns"\\by Frank Buschmann}
\author{André Groß}
\begin{document}
    \maketitle
    \tableofcontents
    
    \part{Architectural Patterns}
    \chapter{From mud to structure}
    
    \section{Layer pattern}
    The Layers architectural pattern helps to structure applications that can
    be decomposed into groups of subtasks in which each group of subtasks is at
    a particular level of abstraction.
    \subsection{Illustration}
    \includegraphics[page=50,clip,trim=5cm 1cm 2.5cm 11.5cm,width=\columnwidth]{buschmann-book.pdf}
    \subsection{Context}
    A large system that requires decomposition
    \subsection{Structure}
    The main structural characteristic of the Layers pattern is that the
    services of Layer $J$ are only used by Layer $J+1$ -- there are no further
    direct dependencies between layers. This structure can be compared with a
    stack, or even an onion. Each individual layer shields all lower layers
    from direct access by higher layers. Each layer con be a complex entity
    consisting of different components.
    \subsection{Benefits}
    \paragraph{Reuse of layers}
    Dramatically reducing of development effort.
    \paragraph{Support for standardization}
    Clearly-defined and commonly-accepted levels of abstraction enable the dev.
    of standardized tasks and interfaces.
    \paragraph{Dependencies are kept local}
    Standardized interfaces between layer usually confine the effect of code
    changes to the layer that is changed.
    \paragraph{Exchangeability}
    Individual layer implementations can be replaced by semantically equivalent
    implementations without too great effort.
    \subsection{Liabilities}
    \paragraph{Cascades of changing behavior}
    When the behavior of a layer changes, it has often the effect, that all
    other layers need to be changed.
    \paragraph{Lower efficiency}
    All relevant data must be transferred through a number of intermediate
    layer, and may be transformed several times.
    \paragraph{Unnecessary work}
    If some services performed by lower layers perform excessive or duplicate
    work not actually required by the higher layer, this has a negative impact
    on performance.
    \paragraph{Establishing the correct granularity of layers}
    A layered architecture with too few layers does not fully exploit this
    patterns potential for reuseability, changeability and portability. On the
    other hand, too many layers introduce unnecessary complexity and overheads
    in the separation of layers and the transformation of arguments and return
    values.
    \subsection{Known Use}
    \paragraph{Virtual Machines} to separate the low levels of hardware and
    operating systems from higher level like the bytecode and software
    functionality.
    \paragraph{APIs} is a classic layer application due to the separation of
    each layer to other layers with defined interfaces like an api is.
    \paragraph{Operating Systems} like Windows NT, its microkernel pattern is
    separated in relaxed layers.
    \paragraph{Information Systems} often use two layers. In the bottom layer
    are the data sinks and in the upper are the funcitonalities.


    \section{Pipes and Filters}
    The Pipes and Filters architectural pattern provides a structure for system
    that process a stream of data Each processing step is encapsulated in a
    filter component. Data is passed through pipes between adjacent filters.
    Recombining filters allows you to build families of related systems.
    \subsection{Illustration}
    \includegraphics[page=76,clip,trim=1cm 7cm 1cm 8cm,width=\columnwidth]{buschmann-book.pdf}
    \subsection{Context}
    Processing data streams.
    \subsection{Structure}
    \paragraph{Filter} components are the processing units of the pipeline. He
    enriches, refines or transforms its input data. A subsequent pipeline
    element pulls output data from the filter. The previous pipeline element
    pushes new input. Most commonly, the filter is active in a loop, pulling
    its input from and push in its output down the pipeline.
    \paragraph{Pipes} denote the connections between filters, between the data
    source and the first filter, and between the last filter and the data sink.
    If two active components are joined, the pipe synchronizes them.
    This synchronization is done with a first-in-first-out buffer.
    \subsection{Benefits}
    \paragraph{No intermediate files necessary, but possible}
    \paragraph{Flexibility by filter exchange.} With their simple interface,
    they are easy exchangeable within a processing pipeline.
    \paragraph{Flexibility by recombination.} This major benefit combined with
    reusability allows to create new processing pipelines by rearranging or
    adding one.
    \paragraph{Reuse of filter components}
    \paragraph{Rapid prototyping of pipelines.}
    Its easy to optimize it incrementally.
    \paragraph{Efficient by parallel processing}
    Each filter is able to compute independent of another.
    \subsection{Liabilities}
    \paragraph{Sharing state information is expensive or inflexible}
    Sharing global data is a problem and in parallel not desirable.
    \paragraph{Efficiency gain by parallel processing is often a illusion}
    Cause of transfer cost, context switching, synchronization and timing.
    \paragraph{Data transformation overhead}
    To gain highest flexibility by using the same data type for all filters
    results in this overhead.
    \paragraph{Error handling}
    In case of an error, the whole pipeline crashes.
    \subsection{Known Use}
    \paragraph{Unix} command shell pipe mechanism.
    \paragraph{CMS Pipeline} CMS is an IBM mainframe operating system 
    extension, to get a unix like behavior.
    \paragraph{Numerical analysis and graphics toolset.}


    \section{Blackboard}
    The Blackboard architectural pattern is useful for problems for which no
    deterministic solution strategies are known. In Blackboard several
    specialized subsystems assemble their knowledge to build a possibly partial
    or approximate solution.
    \subsection{Illustration}
    \includegraphics[page=98,clip,trim=5cm 10.5cm 3cm 3.5cm,width=\columnwidth]{buschmann-book.pdf}
    \subsection{Context}
    An immature domain in which no closed approach to a solution is known or
    feasible.
    \subsection{Structure}
    Divide your System into a component called Blackboard, a collection of
    knowledge sources and a control component.
    \paragraph{The blackboard} is the central data store. Elements of solution
    space and control data are stored here.
    \paragraph{Knowledge sources} are separate, independent subsystems that
    solve specific aspects of the overall problem. None of them can solve the
    task of the system alone.
    \paragraph{The control} component runs a loop that monitors the changes on
    the blackboard and decides what action to take next. It schedules knowledge
    source evaluations and activations according to a knowledge application
    strategy. The bias of this strategy is the data on the blackboard.
    \subsection{Benefits}
    \paragraph{Experimentation is easy}
    \paragraph{Support for changeability and maintainability} due to its
    strictly separated structure. And the blackboard as communication center.
    \paragraph{Reusable knowledge sources}
    \paragraph{Support for fault tolerance and robustness}
    Cause all results are hypotheses, only the most supported survive. So the
    structure has a hight tolerance against noisy data and uncertain solutions.
    \subsection{Liabilities}
    \paragraph{Difficulty of testing} due to the not deterministic algorithm
    used, its not reproducible.
    \paragraph{No good solution is guaranteed}
    \paragraph{Difficulty of establishing a good control strategy.}
    \paragraph{Low Efficiency} computational overhead in rejecting wrong
    hypotheses.
    \paragraph{High development effort} trial and error solutions.
    \paragraph{No support for parallelism}
    \section{Known Use}
    \paragraph{Speech recognition software}
    \paragraph{HASP/SIAP} system to detect enemy submarines.
    \paragraph{CRYSALIS} get three dimensional structure of a protein molecule.
    \paragraph{Tricero} is a system for aircraft activity monitoring.
    \paragraph{Generalization}
    \paragraph{SUS} Software understanding system.


    \chapter{Distributed systems}
    
    \section{Broker}
    The broker architectural pattern can be used to structure distributed
    software systems with decoupled components that interact by remote service
    invocations. A broker component is responsible for coordinating
    communication, such as forwarding requests, as well as for transmitting
    results and exceptions.
    \subsection{Illustration}
    \includegraphics[page=126,clip,trim=3.5cm 11cm 1cm 2.5cm,width=\columnwidth]{buschmann-book.pdf}
    \subsection{Context}
    Your environment is a distributed and possibly heterogeneous system with
    independent cooperating components.
    \subsection{Structure}
    It comprises six types of participating components clients, servers,
    brokers, bridges, client side proxies and server side proxies.
    \paragraph{A server} implements objects that expose their functionality
    through interfaces that consist of operations and attributes.
    \paragraph{Clients} are applications that access the services of at least
    one server. To call remote services, clients forward requests to the
    broker. After an operation has executed they receive responses or
    exceptions from the broker.
    \paragraph{A broker} is a messenger that is responsible for the
    transmission of responses and exceptions back to the client. He offers APIs
    to clients and servers.
    \paragraph{Client side proxies} represent a layer between clients and the
    broker, so that a remote object appears to the client as a local one.
    \paragraph{Server side proxies} are generally analogous to client side ones.
    \paragraph{Bridges} are optional components used for hiding implementation
    details when two brokers interoperate. 
    \subsection{Benefits}
    \paragraph{Location transparency} Clients do not need to know where servers
    are located and servers do not care where clients are.
    \paragraph{Changeability and extensibility of components} means, that if
    servers change but the interface doesnt it has no impact on clients.
    Modifying the broker without its api has also no impact on clients and
    servers and neither a change on the client side has impact on the other
    parts.
    \paragraph{Portability of a broker system.} The broker system hides
    operating system and network system details from clients and servers by
    using indirection layers such as APIs.
    \paragraph{Interoperability between different Broker systems.} Different
    broker systems may interoperate if they understand a common protocol for
    the exchange of messages. This is realized with bridges.
    \paragraph{Reuseability} when building new client applications you can
    often base the functionality of your application on existing services.
    \subsection{Liabilities}
    \paragraph{Restricted efficiency} cause the broker is a bottle neck and
    direct communication is always faster. The indirectional layer of a broker
    enable portability, flexability and changeability in cost of efficiency.
    \paragraph{Lower fault tolerance.} Compared with a non distributed software
    system a broker system may offer lower fault tolerance. Fails a server or a
    broker the applications running on them are unable to continue
    successfully. Redundancy avoids this problem
    \subsection{Liability or benefit?}
    \paragraph{Testing and Debugging} The testing of a client relied on
    services is more robust and easier. But debugging and testing a broker
    system is a tedious job.
    \subsection{Known Use}
    \paragraph{CORBA} object oriented technology for distributing objects on
    heterogeneous systems.
    \paragraph{SOM/DSOM} CORBA-compilant broker system.
    \paragraph{OLE 2.x} is a server interface.
    \paragraph{WWW} Browsers act as brokers and servers play the role of
    service providers.
    \paragraph{ATM-P} telecommunication system.


    
    \chapter{Interactive systems}
    
    \section{Model-View-Controller}
    The MVC architectural pattern divides an interactive application into three
    components. The model contains the core functionality and data. Views
    display information to the user. Controllers handle user input. Views and
    controllers together comprise the user interface. A change-propagation
    mechanism ensures consistency between the user interface and the model.
    \subsection{Illustration}
    \includegraphics[page=148,clip,trim=1cm 8cm 1cm 5cm,width=\columnwidth]{buschmann-book.pdf}
    \subsection{Context}
    Interactive applications with a flexible human computer interface.
    \subsection{Structure}
    \paragraph{The model} component contains the functional core of the
    application. It encapsulates the appropriate data and exports procedures
    that perform application specific processing. Controllers call these
    procedures on behalf of the user. The model also provides functions to
    access its data that are used by view components.
    \paragraph{The view} components present information of the model in
    different ways. Each view defines an update procedure that is activated by
    the change propagation mechanism. Each view creates a suitable controller.
    There is a one to one relationship between view and controller.
    \paragraph{The controller} components accept user input as events. How
    these events are delivered to a controller depends on the user interface
    platform. The event handling could be implemented inside a controller or it
    depends on the state of the model.
    \subsection{Benefits}
    \paragraph{Multiple views of the same model} Cause of the separation from
    view and model
    \paragraph{Synchronized views.} All attached observers are notified of
    changes to the data at the correct time.
    \paragraph{'Pluggable' views and controllers.} 
    \paragraph{Exhangeability of 'look and feel'.} A platform port does not
    affect the functional core.
    \paragraph{Framework potential.}
    \subsection{Liabilities}
    \paragraph{Increased complexity.}
    \paragraph{Potential for excessive number of updates.} A update could
    affect a lot of views to repaint but a repaint must not be done until its
    own model has changed
    \paragraph{Intimate connection between view and controller.} Often there is
    no sharing of controllers between separate views.
    \paragraph{Close coupling of views and controllers to a model.} Like in the
    controller-view relationship.
    \paragraph{Inefficiency of data access in view.} Unnecessary data calls to
    the model (unchanged data).
    \paragraph{Inevitability of change to view and controller when porting.}
    Porting from platform to platform needs change of controllers and view
    \paragraph{Difficulty of using MVC with modern user-interface tools.} 
    \subsection{Known Use}
    \paragraph{Smalltalk}
    \paragraph{MFC} Microsof foundation class library - for developing windows
    applications.
    \paragraph{ET++} is an application framework


    \section{Presentation-Abstraction-Control}
    The Presentation-Abstraction-Control architectural pattern (PAC) defines a
    structure for interactive software systems in the form of a hierarchy of
    cooperating agents. Every agent is responsible for a specific aspect of the
    applications functionality and consists of three components: presentation,
    abstraction and control. This subdivision separates the human-computer
    interaction aspects of the agent from its functional core and its
    communication with other agents.
    \subsection{Illustration}
    Hierarchy:\\
    \includegraphics[width=\columnwidth]{Pac-schema}
    Intermediate level agent:\\
    \includegraphics[page=172,clip,trim=4cm 12.5cm 1cm 3cm,width=\columnwidth]{buschmann-book.pdf}
    \subsection{Context}
    Development of an interactive application with the help of agents
    \subsection{Structure}
    It is a hierarchic pattern with three basic levels.
    \paragraph{Top level PAC agent} provides the global data model.
    \subparagraph{-- The abstraction component} maintain this model.
    \subparagraph{-- The presentation component}
    of it has few responsibilities like user interface
    components common to the whole application. 
    \subparagraph{-- The Control} component allows
    lower level agents to make use of the services of the top level agent,
    mostly to access and manipulate the global data model. It also coordinates
    the hierarchy of agents to ensure correct collaboration and data exchange
    between it self and lower level agents. Additional, it maintains
    information about the interaction of the user with the system.
    \paragraph{Bottom level agents} represent a specific semantic concept of
    the application domain, like a bar chart in a table calculation program.
    \subparagraph{-- The presentation component} presents a specific view of
    the corresponding semantic concept and access to all the functions users
    can apply to it.
    \subparagraph{-- The abstraction component} has similar responsibility as
    the one on top level, maintain agent-specific data, excepts that no other
    agent depend on this data.
    \subparagraph{-- The control component} maintains consistency between the
    abstraction and the presentation components, avoiding direct dependencies
    between them. It also communicates with higher level agents to exchange
    data and events
    \paragraph{Intermediate level agents} fulfill two different roles:
    composition and coordination. Example: A bar chart has bottom agents for
    every bar and a intermediate agent groups them to a bar chart. Or
    coordinate views of the same data like a bar and a line chart of it.
    \subparagraph{-- The abstraction component} maintains the specific data of
    the intermediate level agent.
    \subparagraph{-- The presentation component} implements its user interface.
    \subparagraph{-- The control component} has the responsibilities of
    the bottom level and top level agents together.
    \subsection{Benefits}
    \paragraph{Separation of concerns.} Different semantic concepts in the
    application domain are represented by separate agents.
    \paragraph{Support for change and extension.} Changes to one agent do not
    affect other agents.
    \paragraph{Support for multi tasking.} The agents are native distributed
    and parallel porting affects only the control part of them.
    \subsection{Liabilities}
    \paragraph{Increased system complexity}
    Because every semantic concept needs to be implemented by an own agent.
    \paragraph{Complex control components.} Because they are communication
    mediators between abstraction, presentation and other agents.
    \paragraph{Efficiency} because of a communication overhead between the
    different agents.
    \paragraph{Applicability} The smaller atomic semantic concepts are and the
    greater the similarity of their user interface, the less applicable the
    pattern is.
    \subsection{Known use}
    \paragraph{Network traffic management} to identify bottlenecks in
    telecommunication networks.
    \paragraph{Mobile Robot} to separate the sensors and decissions into well
    separated levels
    
    
    \chapter{Adaptable systems}

    \section{Microkernel}
    The Microkernel architectural pattern applies to software systems that must
    be able to adapt to changing system requirements. It separates a minimal
    functional core from extended functionality and customer-specific parts.
    The microkernel also serves as a socket for plugging in these extensions
    and coordinating their collaboration.
    \subsection{Illustration}
    \includegraphics[page=197,clip,trim=1.5cm 7.5cm 1.5cm 6cm,width=\columnwidth]{buschmann-book.pdf}
    \subsection{Context}
    The development of several applications that use similar programming
    interfaces that build on the same core functionality.
    \subsection{Structure}
    It defines five kinds of participating components:
    \paragraph{The Microkernel} represents the main component of the pattern.
    It implements central services such as communication facilities or resource
    handling. Other components build on all or some of these basic services.
    They do this indirectly by using one or more interfaces that comprise the
    functionality exposed by the microkernel.\\
    In summary, a microkernel implements atomic services, which we refer to as
    mechanisms. These mechanisms serve as a fundamental base on which more
    complex functionality, called policies, are constructed.
    \paragraph{An Internal Server (subsystem)} extends the functionality
    provided by the microkernel. It represents a separate component that
    offers additional functionality. The microkernel invokes the functionality
    of internal servers via service requests. Internal servers can therefore
    encapsulate some dependencies on the underlying hardware or software
    system. For example, device drivers that support specific graphics cards
    are good candidates for internal servers.
    \paragraph{External Servers} expose their functionality by exporting
    interfaces in the same way as the microkernel itself does. Each of these
    external servers runs in a separate process. It receives service requests
    from client applications using the communication facilities provided by the
    microkernel, interprets these requests, executes the appropriate services
    and returns results to its clients. The implementation of services relies
    on microkernel mechanisms, so external servers need to access the
    microkernels programming interfaces.
    \paragraph{A client} is an application that is associated with exactly one
    external server. It only accesses the programming interfaces provided by
    the external server.
    \paragraph{Adapters (emulators)} represent interfaces between clients and
    their external servers and allow clients to access the services of their
    external server in a portable way. They are part of the clients address
    space. A adapter mimics the programming interfaces of that platform and
    they also protect clients from the implementation details of the
    microkernel.
    \subsection{Benefits}
    \paragraph{Portability} Most you don't need to port ext servers or client
    apps if you port the system to a new environment and a port to a new
    hardware environment only requires modification to hardware dependent
    parts.
    \paragraph{Flexibility and Extensibility} for an additional view only
    implement a new ext server. Extending capabilities only requires addition
    or extension of internal servers.
    \paragraph{Separation of policy and mechanism} The microkernel component
    provides all the mechanisms necessary to enable external servers to
    implement their policies.
    \paragraph{Scaleability}
    \paragraph{Reliability} The pattern allows you to run the application on
    more than one machine (availability). If one machine fails you can switch
    to another (fault tolerance).
    \paragraph{Transparency} In distributed system components over a network of
    machines allows this architecture to access other components without
    knowing their location. Communication is hidden from clients by adapter and
    kernel.
    \subsection{Liabilities}
    \paragraph{Performance} Monolithic system. Flexibility and extensibility
    has as always its price in performance.
    \paragraph{Complexity of design and implementation} 


    \section{Reflection}
    The Reflection architectural pattern provides a mechanism for changing
    structure and behavior of software systems dynamically. It supports the
    modification of fundamental aspects, such as type structures and function
    call mechanism. In this pattern, an application is split into two parts.
    A meta level provides information about selected system properties and makes
    the software self-aware. A base level includes the application logic. Its
    implementation builds on the meta level. Changes to information kept in the
    meta level affect subsequent base-level behavior.
    \subsection{Illustration}
    \includegraphics[page=218,clip,trim=1cm 5.5cm 1cm 11cm,width=\columnwidth]{buschmann-book.pdf}
    \subsection{Context}
    Building systems that support their own modification a priori.
    \subsection{Structure}
    \paragraph{Meta level} consists of a set of meta objects. Each meta object
    encapsulates selected information about a single aspect of the structure,
    behavior, or state of the base level.
    \paragraph{Base level} defines the application logic. It uses services
    provided by meta objects and communicates through the meta objects with
    another base level object.
    \paragraph{Meta object protocol.} There is often a protocol, that defines
    the interface between the base and the meta level.
    \subsection{Benefits}
    \paragraph{No explicit modification of source code}
    \paragraph{Changing a software system is easy}
    \paragraph{Support for many kinds of change}
    \subsection{Liabilities}
    \paragraph{Modification at the meta level may cause damage.}
    \paragraph{Increased number of components}
    \paragraph{Lower efficiency.}
    \paragraph{Not all potential changes to the software are supported}
    \paragraph{Not all languages support reflection.}



%%%%%%%%%%%%%%%%%%%%%%%%%%%%%%%%%%%%%%%%%%%%%%%%%%%%%%%%%%%%%%%%%%%%%%%%%%%%%%%
    \part{Design patterns}
    \chapter{Structural decomposition}

    \section{Whole-Part}
    The Whole-Part design pattern helps with the aggregation of components that
    together form a semantic unit. An aggregate component, the Whole,
    encapsulates its constituent components, the Parts, organizes their
    collaboration and provides a common interface to its functionality. Direct
    access to the Parts is not possible.
    \subsection{Illustration}
    \includegraphics[page=248,clip,trim=1.5cm 12cm 1.5cm 3cm,width=\columnwidth]{buschmann-book.pdf}
    \subsection{Context}
    Implementing aggregate objects.
    \subsection{Structure}
    \paragraph{The whole} object represents an aggregation of smaller objects,
    which we call Parts. It forms a semantic grouping of its Parts in that it
    coordinates and organizes their collaboration. For this purpose, the Whole
    uses the functionality of Part objects for implementing services.
    \paragraph{Each Part} object is embedded in exactly one Whole. Two or more
    Choles cannot share the same Part. Each Part is created and destroyed
    within the life span of its Whole.
    \subsection{Benefits}
    \paragraph{Changeability of Parts} modify internal structure without impact
    on clients.
    \paragraph{Separation of concerns.} Each concern is implemented by a part.
    \paragraph{Reuseability} Parts can be reused in other aggregate objects.
    And Wholes can be reused with scattered parts.
    \subsection{Liabilities}
    \paragraph{Lower efficiency through indirection.}
    \paragraph{Complexity of decomposition into Parts.} This pattern is often
    hard to find. Especial when a bottom up approach is applied.
    
    
    \chapter{Organisation of work}

    \section{Master-Slave}
    The Master-Slave design pattern supports fault tolerance, parallel
    computation and computational accuracy. A master component distributes work
    to identical slave components and computes a final result from the results
    these slaves return.
    \subsection{Illustration}
    \includegraphics[page=275,clip,trim=4.5cm 7cm 1cm 9.5cm,width=\columnwidth]{buschmann-book.pdf}
    \subsection{Context}
    Partitioning work into semantically-identical sub-tasks.
    \subsection{Structure}
    \paragraph{The master component} provides a service that can be solved by
    applying the 'divide and conquer' principle. It offers an interface that
    allows clients to access this service. Internally the master implements
    functions for partitioning work into several equal sub tasks, starting and
    controlling their processing and computing a final result from all the
    results obtained.
    \paragraph{Tho slave component} provides a sub service that can process the
    sub tasks defined by the master. Within a master slave structure there are
    at least two instances of the slave component connected to the master.
    \subsection{Benefits}
    \paragraph{Exchangeability and extensibility} Due to the abstract slave
    class.
    \paragraph{Separation of concerns.}
    \paragraph{Efficiency.}
    \subsection{Liabilities}
    \paragraph{Feasibility.}
    \paragraph{Machine dependency.}
    \paragraph{Hard to implement}
    \paragraph{Portability}
    
    
    \chapter{Access Control}

    \section{Proxy}
    The Proxy design pattern makes the clients of a component communicate with
    a representative rather than to the component itself. Introducing such a
    placeholder can serve many purposes, including enhanced efficiency, easier
    access and protection from unauthorized access.
    \subsection{Illustration}
    \includegraphics[page=285,clip,trim=4.5cm 12cm 1.5cm 3cm,width=\columnwidth]{buschmann-book.pdf}
    \subsection{Context}
    A client needs access to the services of another component. Direct access
    is technically possible, but may not be the best approach.
    \subsection{Structure}
    \paragraph{The original} implements a particular service.
    \paragraph{The client} is responsible for a task that needs a service. So 
    he invokes the functionality of the original indirect by accessing the 
    proxy. 
    \paragraph{The proxy} offers the same interface as the original and ensures
    correct access to the original. 
    \paragraph{The abstract original} provides the interface implemented by the
    proxy and the original.
    \subsection{Benefits}
    \paragraph{Enhanced efficiency and lower cost.} It helps to implement a
    load on demand strategy. To avoid unnecessary loads from slow services with
    caching.
    \paragraph{Decoupling clients from the location of server components.}
    \paragraph{Separation of housekeeping code from functionality.}
    \subsection{Liabilities}
    \paragraph{Less efficient due to indirection.}
    \paragraph{Overkill via sophisticated strategies.} Example is caching of
    highly dynamic applications.
    
    
    \chapter{Management}

    \section{Command Processor}
    The Command processor design pattern separates the request for a service
    from its execution. A command processor component manages requests as
    separate objects, schedules their execution and provides additional
    services such as the storing of request objects for later undo
    \subsection{Illustration}
    \includegraphics[page=299,clip,trim=1.5cm .5cm 1cm 13.5cm,width=\columnwidth]{buschmann-book.pdf}
    \subsection{Context}
    Applications that need flexible and extensible user interfaces or
    applications that provide services related to the execution of user
    functions such as scheduling or undo.
    \subsection{Structure}
    \paragraph{The abstract command} component
    \paragraph{The controller}
    \paragraph{The command processor}
    \paragraph{The supplier} components
    \subsection{Benefits}
    \paragraph{Flexibility in the way requests are activated.}
    \paragraph{Flexibility in the number and functionality of requests}
    \paragraph{Programming execution related services.}
    \paragraph{Testability at application level.}
    \paragraph{Concurrency}
    \subsection{Liabilities}
    \paragraph{Efficiency loss}
    \paragraph{Potential for an excessive number of command classes.}
    \paragraph{Complexity} in acquiring command parameters.


    \section{View Handler}
    The View Handler design pattern helps to manage all views that a software
    system provides. A view handler component allows clients to open manipulate
    and dispose of views. It also coordinates dependencies between views and
    organizes their update.
    \subsection{Illustration}
    \includegraphics[page=314,clip,trim=2cm 3cm 2cm 8.5cm,width=\columnwidth]{buschmann-book.pdf}
    \subsection{Context}
    A software system that provides multiple views of application specific data
    or that supports working with multiple documents.
    \subsection{Structure}
    \paragraph{The view handler} is the central component. It is responsible
    for opening new views and coordinates them. Clients can specify the view 
    they want.
    \paragraph{An abstract view} component defines an interface that is common
    to all views.
    \paragraph{Specific view} components are derived from the abstract view and
    implements its interface.
    \paragraph{Supplier} components provide the data that is displayed by view
    components.
    \subsection{Benefits}
    \paragraph{Uniform handling of views}
    \paragraph{Extensibility and changeability of views}
    \paragraph{Application specific view coordination}
    \subsection{Liabilities}
    \paragraph{Restricted applicability}
    \paragraph{Efficiency}
    
    
    \chapter{Communication}

    \section{Forwarder-Reciever}
    The Forwarder-Reciever design pattern provides transparent interprocess
    communication for software systems with a peer-to-peer interaction model.
    It introduces forwarders and receivers to decouple peers from the
    underlying communication mechanisms.
    \subsection{Illustration}
    \includegraphics[page=330,clip,trim=1.5cm 9.5cm 1cm 3.5cm,width=\columnwidth]{buschmann-book.pdf}
    \subsection{Context}
    Peer-to-Peer Communication.
    \subsection{Structure}
    \paragraph{Peer} components are responsible for application tasks. To carry
    out their tasks peers need to communicate with other peers. It uses a
    forwarder to send messages to other peers and a receiver to receive
    messages.
    \paragraph{Forwarder} components send messages across process boundaries.
    \paragraph{Receiver} components are responsible for receiving messages that
    were sent by a forwarder from a peer.
    \subsection{Benefits}
    \paragraph{Efficient inter process communication}
    \paragraph{Encapsulation of IPC facilities}
    \subsection{Liabilities}
    \paragraph{No support for flexible reconfiguration of components.}


    \section{Client-Dispatcher-Server}
    The client dispatcher server design pattern introduces an intermediate
    layer between clients and servers, the dispatcher component. It provides
    location transparency by means of a name service and hides the details of
    the establishment of the communication connection between clients and
    servers.
    \subsection{Illustration}
    \includegraphics[page=345,clip,trim=2cm 12.5cm 1cm 3cm,width=\columnwidth]{buschmann-book.pdf}
    \subsection{Context}
    A software system integrating a set of distributed servers, with the
    servers running locally or distributed over a network.
    \subsection{Structure}
    \paragraph{A client} uses the server to do its task. Before sending a
    request to a server, he asks the dispatcher for a communication channel
    that he will use for the communication.
    \paragraph{A server} provides a set of operations to clients. He is
    registered in the dispatcher.
    \paragraph{The dispatcher} offers functionality for establishing
    communication channels between clients and servers.
    \subsection{Benefits}
    \paragraph{Exchangeability of servers}
    \paragraph{Location and migration transparency}
    \paragraph{Reconfiguration}
    \paragraph{Fault tolerance}
    \subsection{Liabilities}
    \paragraph{Lower efficiency through indirection and explicit connection
        establishment}
    \paragraph{Sensitivity to change in the interfaces of the dispatcher
        component}



    \section{Publisher-Subscriber}
    The Publisher-Subscriber design pattern helps to keep the state of
    cooperating components synchronize. To achieve this it enables one-way
    propagation of change: one publisher notifies any number of subscribers
    about changes to its state.
    \subsection{Illustration}
    \includegraphics[page=361,clip,trim=1.5cm 14.5cm 1cm 2cm,width=\columnwidth]{buschmann-book.pdf}
    \subsection{Annotation}
    This pattern is not well documented like the other patterns. So I will skip
    the explanatory parts of this pattern.

\end{document}
